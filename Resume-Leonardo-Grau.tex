\documentclass{article}

\usepackage{fontspec}
\setmainfont{Inter}

\usepackage[margin=1.5cm]{geometry}

\usepackage[hidelinks]{hyperref}

\usepackage{tabularx}

\usepackage{enumitem}
\setlength{\parindent}{0pt}



\begin{document}
\begin{minipage}{0.40\textwidth}
{
\noindent
\large \textbf {Leonardo Manuel Grau Vargas}\\
\small Software Engineer Graduate
}
\end{minipage}
\hfill
\begin{minipage}{0.36\textwidth}
{
Email: leonardo.grau@outlook.com.pe\\
Github: \href{https://github.com/LeoGrau}{LeoGrau}\\
LinkedIn: \href{https://github.com/LeoGrau}{Leonardo Manuel Grau Vargas}\\
}
\end{minipage}


\begin{center}
\section*{Perfil Profesional}
\end{center}
Desarrollador Full-Stack Junior con experiencia en el desarrollo de aplicaciones web, integrando frontend y backend. Trabajo con tecnologías modernas para la construcción de APIs REST y aplicaciones SPA, participando en el diseño, desarrollo e integración de soluciones. Enfoque en aprendizaje continuo, buenas prácticas y resolución de problemas.
\vspace{6pt}

\section*{Educación}
\begin{tabularx}{\textwidth}{X r}
\textbf{Universidad Peruana de Ciencias Aplicadas} & \textbf{2020--2025} \\[2pt]
Ingeniería de Software & Lima, Perú \\
\end{tabularx}


\section*{Experiencia Laboral}
\begin{tabularx}{\textwidth}{X r}
\textbf{Outlier} & \textbf{Remote -- Latin America} \\
\textbf{Desarrollador Freelancer} & \textbf{enero 2025--diciembre 2025} \\
\end{tabularx}

\vspace{-12pt}

\begin{tabularx}{\textwidth}{X r}
\multicolumn{2}{X}{
\begin{itemize}[itemsep=2pt]
\item Revisión y mejora de código generado por modelos de IA, asegurando calidad, claridad y buenas prácticas de desarrollo.
  
\item Corrección y optimización de aplicaciones en Python y React, mejorando funcionamiento, estructura y experiencia de usuario.
  
\item Validación de cambios y migraciones en backend con Spring, detectando errores y reduciendo riesgos en producción.
  
\item Evaluación de respuestas técnicas de agentes de IA, aportando feedback para mejorar precisión y consistencia técnica.
\end{itemize}
}
\end{tabularx}

\vspace{6pt}

\begin{tabularx}{\textwidth}{X r}
\textbf{Valticore} & \textbf{Lima, Perú} \\
\textbf{Desarrollador Móvil y Web Fullstack} & \textbf{julio 2024--diciembre 2024} \\
\end{tabularx}

\vspace{-12pt}

\begin{tabularx}{\textwidth}{X r}
\multicolumn{2}{X}{
\begin{itemize}[itemsep=2pt]
\item Desarrollo de una aplicación Android para seguimiento de flota en tiempo real, integrando Google Maps API y comunicación en vivo.
  
\item Implementación de backend en .NET 6 con Entity Framework Core y LINQ, utilizando SQL Server y arquitectura hexagonal.
  
\item Integración frontend--backend para gestión de roles y permisos en aplicaciones web.
  
\item Trabajo en equipo bajo metodologías ágiles (Scrum/Kanban) e integración de control de versiones y CI/CD en Bitbucket.
\end{itemize}
}
\end{tabularx}

\vspace{6pt}

\begin{tabularx}{\textwidth}{X r}
\textbf{Controlware} & \textbf{Lima, Perú} \\
\textbf{Desarrollador Web Fullstack} & \textbf{julio 2023--octubre 2024} \\
\end{tabularx}

\vspace{-12pt}

\begin{tabularx}{\textwidth}{X r}
\multicolumn{2}{X}{
\begin{itemize}[itemsep=2pt]
\item Diseño y desarrollo de APIs REST en .NET 8, aplicando patrones de arquitectura para asegurar escalabilidad y mantenibilidad.
  
\item Administración y optimización de bases de datos PostgreSQL, mejorando el rendimiento de consultas y acceso a datos.
  
\item Desarrollo de aplicaciones SPA con Vue 3 (Composition API) y PrimeVue, siguiendo una arquitectura modular basada en componentes.
  
\item Containerización y despliegue de aplicaciones con Docker en servidores Linux (Ubuntu).
\end{itemize}
}
\end{tabularx}


\section*{Habilidades}

\begin{itemize}[itemsep=3pt]
  \item \textbf{Desarrollo Frontend:} Desarrollo de interfaces web y aplicaciones SPA, utilizando frameworks modernos como Vue.js y Angular, con consumo de APIs REST.

  \item \textbf{Desarrollo Backend:} Desarrollo de APIs REST utilizando .NET y ASP.NET Core, con acceso a datos mediante Entity Framework Core y LINQ.

  \item \textbf{Bases de Datos:} Uso de bases de datos relacionales (SQL Server, PostgreSQL) para almacenamiento y consulta de datos.

  \item \textbf{Herramientas:} Git, Docker, Linux (Ubuntu y Fedora), uso básico de CI/CD.

  \item \textbf{Lenguajes:} C\#, JavaScript, TypeScript, Java, Python.
  \item \textbf{Herramientas de productividad y documentación:}  Microsoft Word, Excel, PowerPoint, MarkText, LaTex
\end{itemize}



\section*{Proyectos Relevantes}

\begin{itemize}[itemsep=6pt]

\item \textbf{WiraChain (Proyecto de Tesis)} \\
Desarrollo de una plataforma web para la interoperabilidad de historiales clínicos en centros de salud, basada en el estándar HL7 FHIR y tecnología Blockchain. Implementación de servicios backend, modelado de datos clínicos y mecanismos de seguridad para garantizar la integridad y trazabilidad de la información médica.  
\textit{Artículo aceptado para publicación en revista Q1 (Frontiers in Public Health, 2026).}

\item \textbf{NotixApp} \\
Desarrollo de una aplicación web full-stack para la gestión de notas, implementando autenticación JWT, backend en .NET y frontend en Vue.js. Containerización con Docker para facilitar el despliegue y la portabilidad.

\end{itemize}

\section*{Formación Complementaria}

\begin{itemize}
  \item \textbf{Python for Everybody Specialization} \\
  Coursera - University of Michigan - Octubre 2024

  \item \textbf{Blockchain Specialization} \\
  Coursera - University at Buffalo - Octubre 2024

  \item \textbf{Desarrollo HTML, CSS y Javascript} \\
  Coursera - John Hopkins University - Octubre 2024

 \item \textbf{Google ISC IT Support} \\
  Coursera - Google Cloud - Octubre 2024
\end{itemize}

\end{document}
