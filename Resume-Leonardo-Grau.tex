\documentclass{article}

\usepackage{fontspec}
\setmainfont{Inter}

\usepackage[margin=1.5cm]{geometry}

\usepackage[hidelinks]{hyperref}

\usepackage{tabularx}

\usepackage{enumitem}
\setlength{\parindent}{0pt}



\begin{document}
\begin{minipage}{0.50\textwidth}
  {
    \noindent
    \large \textbf {Leonardo Manuel Grau Vargas}\\
    \small Full-Stack Software Engineer (.NET • Vue • Cloud • APIs)
  }
\end{minipage}
\hfill
\begin{minipage}{0.36\textwidth}
  {
    Email: leonardo.grau@outlook.com.pe\\
    Github: \href{https://github.com/LeoGrau}{LeoGrau}\\
    LinkedIn: \href{www.linkedin.com/in/leonardo-manuel-grau-vargas-35811a1a7}{Leonardo Manuel Grau Vargas}\\
  }
\end{minipage}


\begin{center}
  \section*{Perfil Profesional}
\end{center}
Ingeniero de Software Full-Stack con experiencia en el desarrollo de aplicaciones web, APIs REST y aplicaciones SPA, integrando frontend y backend en entornos productivos. He participado en el diseño, implementación y despliegue de soluciones utilizando .NET, bases de datos relacionales y contenedores Docker en servidores Linux. Experiencia trabajando con arquitecturas modulares, buenas prácticas de desarrollo y procesos de integración continua. Enfoque en rendimiento, mantenibilidad e interoperabilidad de sistemas.
\vspace{9pt}

\section*{Educación}
\begin{tabularx}{\textwidth}{X r}
  \textbf{Universidad Peruana de Ciencias Aplicadas} & \textbf{2020--2025} \\[2pt]
  Ingeniería de Software                             & Lima, Perú          \\
\end{tabularx}


\section*{Experiencia Laboral}

\begin{tabularx}{\textwidth}{X r}
  \textbf{Controlware}                 & \textbf{Lima, Perú}               \\
  \textbf{Desarrollador Web Fullstack} & \textbf{julio 2023--octubre 2024} \\
\end{tabularx}

\vspace{-12pt}

\begin{tabularx}{\textwidth}{X r}
  \multicolumn{2}{X}{
    \begin{itemize}[itemsep=2pt]
      \item Diseño y desarrollo de APIs REST en .NET 6, aplicando patrones de arquitectura para garantizar escalabilidad y mantenibilidad de los servicios.

      \item Optimización de consultas y administración de bases de datos PostgreSQL, mejorando el rendimiento de acceso a información en entornos productivos.

      \item Desarrollo de aplicaciones SPA con Vue 3 (Composition API) y PrimeVue, implementando arquitectura modular basada en componentes.

      \item Containerización y despliegue de aplicaciones medianRevisión y refactorización de código generado por IA, asegurando calidad y buenas prácticas. Debugging y optimización de aplicaciones en Python y React. Validación de migraciones backend en Spring Boot y detección de errores críticos previo a despliegues. Evaluación técnica de respuestas de agentes de IA, proporcionando feedback para mejorar precisión y consistencia de soluciones de software.te Docker en servidores Linux (Ubuntu).

    \end{itemize}
  }
\end{tabularx}
\vspace{9pt}

\begin{tabularx}{\textwidth}{X r}
  \textbf{Valticore}                           & \textbf{Lima, Perú}                 \\
  \textbf{Desarrollador Móvil y Web Fullstack} & \textbf{julio 2024--diciembre 2024} \\
\end{tabularx}

\vspace{-12pt}

\begin{tabularx}{\textwidth}{X r}
  \multicolumn{2}{X}{
    \begin{itemize}[itemsep=2pt]
      \item Desarrollo de aplicación Android para seguimiento de flota en tiempo real, integrando Google Maps API y comunicación en vivo para monitoreo operativo.

      \item Implementación de servicios backend en .NET 6 con Entity Framework Core y LINQ, bajo arquitectura hexagonal.

      \item Diseño e integración de módulos de autenticación, roles y permisos entre frontend y backend.

      \item Participación en flujos de trabajo ágiles (Scrum/Kanban) e integración de control de versiones y pipelines CI/CD en Bitbucket.

    \end{itemize}
  }
\end{tabularx}
\newpage

\begin{tabularx}{\textwidth}{X r}
  \textbf{Outlier}                  & \textbf{Remote -- Latin America}    \\
  \textbf{Desarrollador Freelancer} & \textbf{junio 2025--noviembre 2025} \\
\end{tabularx}

\vspace{-12pt}

\begin{tabularx}{\textwidth}{X r}
  \multicolumn{2}{X}{
    \begin{itemize}[itemsep=2pt]
      \item Revisión técnica y refactorización de código generado por modelos de IA, asegurando calidad, mantenibilidad y cumplimiento de buenas prácticas de desarrollo.

      \item Debugging y optimización de aplicaciones en Python y React, corrigiendo defectos funcionales y mejorando la estructura del código.

      \item Validación de migraciones y cambios backend en entornos Spring Boot, identificando errores críticos antes de despliegues productivos.

      \item Evaluación técnica de respuestas generadas por agentes de IA, proporcionando feedback especializado para mejorar precisión y consistencia de soluciones de software.

    \end{itemize}
  }
\end{tabularx}
\vspace{6pt}


\section*{Habilidades}

\begin{itemize}[itemsep=3pt]
  \item \textbf{Frontend:} Desarrollo de aplicaciones SPA con Vue.js (Composition API) y Angular, consumo e integración de APIs REST, gestión de estado y arquitectura basada en componentes.

  \item \textbf{Backend:} Desarrollo de APIs REST con .NET y ASP.NET Core, implementación de lógica de negocio, autenticación, autorización y acceso a datos con Entity Framework Core y LINQ.

  \item \textbf{Bases de Datos:} Diseño, administración y optimización de bases de datos relacionales (SQL Server, PostgreSQL).

  \item \textbf{DevOps \& Infraestructura:} Containerización con Docker, despliegue en servidores Linux (Ubuntu/Fedora), integración continua básica (CI/CD).

  \item \textbf{Arquitectura:} Arquitectura hexagonal, diseño modular de servicios y buenas prácticas en desarrollo de software.

  \item \textbf{Lenguajes:} C\#, JavaScript, TypeScript, Java, Python.
\end{itemize}



\section*{Proyectos Relevantes}

\begin{itemize}[itemsep=6pt]

  \item \textbf{WiraChain – Plataforma de Interoperabilidad Clínica basada en Blockchain} \\

        Diseño e implementación de una plataforma distribuida para el intercambio seguro de historiales clínicos entre centros de salud, basada en el estándar HL7 FHIR y tecnología Blockchain. Se desarrollaron servicios backend en .NET para la gestión de lógica de negocio, validación de datos clínicos e integración con la red blockchain.

        Se implementó una red blockchain privada utilizando Hyperledger Besu mediante smart contracts para garantizar integridad, trazabilidad e inmutabilidad de la información médica, integrando además un servidor FHIR basado en Java para la interoperabilidad de recursos clínicos bajo estándares internacionales.

        El sistema incluye un frontend SPA en React para la visualización y gestión de historiales clínicos, bajo una arquitectura modular orientada a interoperabilidad entre sistemas heterogéneos del sector salud.

        \textbf{Stack:} .NET, React, Hyperledger Besu, HL7 FHIR (Java), PostgreSQL, Docker. \\
        \textit{Artículo científico aceptado para publicación en revista Q1 (Frontiers in Public Health, 2026).}

  \item \textbf{NotixApp – Sistema Web Full-Stack de Gestión de Notas} \\

        Diseño y desarrollo de una aplicación web end-to-end para la gestión de notas, implementando autenticación segura basada en JWT y control de acceso por usuario. Se diseñaron e implementaron APIs REST en .NET para operaciones CRUD, validación de datos y lógica de negocio, integradas con un frontend SPA en Vue.js para interacción dinámica con los servicios backend.

        La solución fue integrada con base de datos relacional para persistencia de información y containerizada mediante Docker, permitiendo su despliegue portable en entornos Linux.

        \textbf{Stack:} .NET, ASP.NET Core, Vue.js, SQL Server/PostgreSQL, Docker, JWT Authentication.

\end{itemize}

\section*{Idiomas}

\begin{itemize}[itemsep=2pt]
  \item \textbf{Español:} Nativo.
  \item \textbf{Inglés:} Avanzado (comunicación técnica profesional).
\end{itemize}


\section*{Formación Complementaria}

\begin{itemize}
  \item \textbf{Blockchain Specialization} – University at Buffalo (Coursera), 2024.

  \item \textbf{Python for Everybody Specialization} – University of Michigan (Coursera), 2024.

  \item \textbf{Google IT Support Professional Certificate} – Google (Coursera), 2024.

  \item \textbf{HTML, CSS \& JavaScript for Web Developers} – Johns Hopkins University (Coursera), 2024.
\end{itemize}

\end{document}
